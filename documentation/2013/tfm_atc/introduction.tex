
\begin{savequote}[50mm]
I think that most of us, anyway, read these stories that we know are not "true" because we're hungry for another kind of truth: the mythic truth about human nature in general, the particular truth about those life-communities that define our own identity, and the most specific truth of all: our own self-story. Fiction, because it is not about someone who lived in the real world, always has the possibility of being about oneself. 
\qauthor{Orson Scott Card}
\end{savequote}


\chapter{Introducción}

%Las subsecciones del presente capítulo tienen como objetivo contextualizar la realización del proyecto MADE. . La subsección \ref{sec:motivacion} presenta la necesidad incipiente de una herramienta que facilite la labor de generación de tramas, ante un mercado del entretenimiento voraz que pide mundos abiertos cada vez más realistas. La subsección \ref{objetivos} presenta de manera somera los principales objetivos de MADE.

La herramienta MADE nace ante la necesidad de apoyar y automatizar la labor de diseño de tramas secundarias en ficción interactiva. Para poder defender la existencia de dicha necesidad, es necesario contextualizar el proyecto en los marcos tecnológico y social actuales:

La escritura de un libro o un guión cinematográfico es una tarea de gran complejidad que incluso a una persona talentosa puede llevarle toda la vida. Existen multitud de parámetros formales para medir la calidad de un escrito, pero a la hora de valorar la trama entran en juego los gustos y la percepción personal del lector. En términos generales, los gurús de la escritura de guiones recalcan que no existe una fórmula para hacer una buena película; en su lugar, existen ciertas prácticas que funcionan en general si se aplican correctamente, por ejemplo, la división en 3 actos, los cliffhangers, loc conflictos o el seguimiento estricto de los tiempos, por citar algunos.

Sin embargo, un nuevo ámbito cinematográfico está emergiendo en los últimos años: los videojuegos. Ciertas familias de videojuegos pueden sumergir al jugador en una historia guionizada absolutamente inmersiva, e incluso permitir cierta flexibilidad en cuanto a los resultados de la historia. En este sentido, los sandboxes representan el máximo exponente de libertad en un videojuego, permitiendo al usuario interactuar con todos los personajes, animales, principales objetos y terreno, y llevándole a través de tramas secunadarias, en ocasiones auto-generadas con la finalidad de que siempre sean diferentes para el usuario y permitan descubrir zonas ocultas del mapa.

A la hora de convertir un guión o escrito clásico a un sandbox hay que tener en cuenta que la historia debe permitir libertad de actuación, mostrar coherencia y fidelidad con lo que el usuario considera la realidad dentro del contexto del juego. Los retos tecnológicos ante este tipo de entretenimiento emergente ofrecen grandes posibilidad en diferentes campos de investigación:

\begin{itemize}
\item motores gráficos: modelado, renderización, iluminación, shaders, etc
\item motores de física: colisiones, movimiento, sistemas de partículas, etc
\item motores de sonido: Mejor calidad, efectos, texto a voz
\item inteligencia artificial: Comportamiento de los personajes, estrategia, reconocimiento de voz, de caras, auto-ajuste de la dificultad, etc
\item nuevas interfaces: gestuales, por voz, nuevos sensores, etc
\end{itemize}

Sin embargo, tanto el desarrollo de los personajes secundarios y extras como la coherencia y profundidad de sus tramas, no son considerados como campos tecnológicos de investigación relevantes para las compañias desarrolladoras, y este aspecto queda demostrado en los videojuegos de última generación.

Con la llegada de los sandboxes un jugador puede cruzarse con miles de personajes a lo largo de una partida, sin embargo, diseñar tramas para tal número de personajes sería inviable. En la prática, los personajes cumplen un perfil y realizan ciertas acciones básicas cuando el usuario está cerca, pero carecen de frases, motivaciones, emociones, objetivos e interrelaciones. Como consecuencia, el mundo virtual representado sigue siendo inanimado, irreal y mecánico. Por ello, bajo el punto de vista del autor, la guinización y las tramas de personajes secundarios y extras sí es un reto tecnológico. MADE, pretende abordar ese problema mediante un sistema auto-organizativo que modela la población como un conjunto de agentes que interaccionan entre sí, y un sistema bioinspirado que permite optimizar el comportamiento del sistema auto-organizativo para que además sea interesante para el jugador final, permitiéndole además que cada nueva partida pueda tener un mundo diferente a la anterior.





% TODO
% ¿Qué es una buena historia?
% Cómo se convierte una buena historia en una historia interactiva
% Qué características tiene una historia interactiva
% Qué retos tecnológicos implican las historias interactivas
% Centrar la pregunta sobre el gran olvidado, las tramas
% ¿Por qué es el gran olvidado?
% Problemas en mundos abiertos
% Es inviable en tiempo
% En consecuencia, los personajes son planos
% repercute en el mundo virtual, haciéndolo plano
% Nace un nuevo reto tecnológico: Diseñar tramas secundarias


La figura \ref{fig:panorama_ficcion} muestra de manera esquematizada el panorama actual de la ficción interactiva.



\InsertFig{panorama_ficcion.pdf}{fig:panorama_ficcion}{Panorama actual de la ficción interactiva.}{See Macro.tex for a detailed explanation of the InsertFig function}{1}{}


\section{Objetivos de MADE}
\label{objetivos}

MADE es un motor de drama artificial masivo para personajes no jugadores, es decir, una herramienta que permite crear mundos coherentes, con personajes y tramas interesantes, de manera automática.
